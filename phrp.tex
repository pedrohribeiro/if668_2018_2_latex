\documentclass[10pt]{article}
\usepackage{hyperref}
\usepackage[utf8]{inputenc}
\usepackage[brazil]{babel}

\title{IF680 - Processamento Gráfico}
\author{Pedro Henrique Ribeiro Pereira}
\date{Outubro 2018}

\usepackage{natbib}
\usepackage{graphicx}

\begin{document}

\maketitle

\begin{figure}[h!]
\centering
\includegraphics[scale=.35]{PG.jpeg}
\caption{Processamento Gráfico}
\end{figure}

\section{Introdução}
A Computação Gráfica é uma área da Ciência da Computação que se dedica
ao estudo e desenvolvimento de técnicas e algoritmos para a geração (síntese) de imagens
através do computador. Ela pode ser definida como o conjunto de técnicas que são utilizados para transformar dados em em conteúdo gráfico. Diferentes setores do mercado utilizam a Computação Gráfica em seus processos de planejamento e produção. 

Essa transformação pode ser feita através de softwares, programas feitos para computador, e vai desde de um desenho animado até o projeto de uma construção. Por esse motivo, essa é uma área em constante expansão que oferece diferentes tipos de oportunidade.
\citep{NEWS}

\section{Relevância}
Grandes empresas utilizam da Computação Gráfica para desenvolvimento de desenhos animados, na criação de jogos onde o profissional necessita atuar na criação e modelagem dos personagens.

Um ponto positivo sobre essa área é que você pode fazer simulações de espaços para treinamentos e aperfeiçoamentos de técnicas. Por exemplo: têm sido desenvolvidos sistemas de simulação para auxiliar no
treinamento de cirurgias endoscópicas. Outros tipos de simuladores são usados para treinamento
de pilotos e para auxiliar na tomada de decisões na área do direito (por exemplo, para
reconstituir a cena de um crime).

\citep{PUCRS}

\section{Relação com outras disciplinas}
\begin{table}[h]
\begin{tabular}{|c|c|}
\hline
\textbf{CADEIRA}                                                                              & \textbf{RELAÇÃO}                                                                                                                                                                                                                                                                                                                                                                       \\ \hline
\begin{tabular}[c]{@{}c@{}}MA026 -\\ CÁLCULO\\ DIFERENCIAL E\\ INTEGRAL 1\end{tabular}        & \begin{tabular}[c]{@{}c@{}}Cálculo 1 é uma cadeira que se é paga no primeiro período\\ de Ciências da computação.  Ela proporciona para o estudante\\ um avanço no seu raciocínio, através de ferramentas\\ como o estudo de Derivadas e Integrais.  Ferramentas essas\\ que serão usadas na cadeira de Processamento Gráfico.\end{tabular}                                            \\ \hline
\begin{tabular}[c]{@{}c@{}}MA531 -\\ ÁLGEBRA VETO -\\ RIAL LINEAR\\ P/COMPUTAÇÃO\end{tabular} & \begin{tabular}[c]{@{}c@{}}Está estritamente relacionada, pois, a principal ferramenta\\ utilizada no aprendizado da álgebra é o estudo de vetores\\ e matrizes.  Que são a fonte principal do Processamento\\ Gráfico.  Cadeira que também é estudada no primeiro período \\ de Ciências da Computação.  Ambas matérias são pré-requisitos\\  para Processamento Gráfico\end{tabular} \\ \hline
\end{tabular}
\end{table}

\citep{CADEIRAS}

\bibliographystyle{plain}
\bibliography{references}
\end{document}
